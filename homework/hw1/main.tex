\documentclass{article}
\usepackage[margin=1.5in]{geometry} % Please keep the margins at 1.5 so that there is space for grader comments.
\usepackage{amsmath,amsthm,amssymb,hyperref}
\usepackage{amsmath,amscd}
\usepackage[all]{xy}
\usepackage{tikz-cd}
\usepackage[all]{xy}
\usepackage{mathrsfs}
\usepackage{amsthm}
\usepackage{amssymb}
\usepackage{amsmath}
\usepackage{amsfonts}      
\usepackage{nicefrac}
\usepackage{extarrows}
\usepackage{blindtext}
\usepackage[english]{babel}
\usepackage{setspace}
\usepackage{hyperref}
\usepackage{subfigure}
\usepackage{subcaption}
\usepackage{color}
\usepackage{enumerate}
\usepackage{mathtools}
 \usepackage{empheq}
 
 \usepackage[fontset=ubuntu]{ctex}
 
\newcommand{\R}{\mathbf{R}}
\newcommand{\Z}{\mathbf{Z}}
\newcommand{\N}{\mathbf{N}}
\newcommand{\Q}{\mathbf{Q}}

\newenvironment{theorem}[2][Theorem]{\begin{trivlist}
\item[\hskip \labelsep {\bfseries #1}\hskip \labelsep {\bfseries #2.}]}{\end{trivlist}}
\newenvironment{lemma}[2][Lemma]{\begin{trivlist}
\item[\hskip \labelsep {\bfseries #1}\hskip \labelsep {\bfseries #2.}]}{\end{trivlist}}
\newenvironment{claim}[2][Claim]{\begin{trivlist}
\item[\hskip \labelsep {\bfseries #1}\hskip \labelsep {\bfseries #2.}]}{\end{trivlist}}
\newenvironment{problem}[2][Problem]{\begin{trivlist}
\item[\hskip \labelsep {\bfseries #1}\hskip \labelsep {\bfseries #2.}]}{\end{trivlist}}
\newenvironment{proposition}[2][Proposition]{\begin{trivlist}
\item[\hskip \labelsep {\bfseries #1}\hskip \labelsep {\bfseries #2.}]}{\end{trivlist}}
\newenvironment{corollary}[2][Corollary]{\begin{trivlist}
\item[\hskip \labelsep {\bfseries #1}\hskip \labelsep {\bfseries #2.}]}{\end{trivlist}}

\newenvironment{solution}{\begin{proof}[Solution]}{\end{proof}}

\begin{document}

\large

\begin{center}
{\Large 作业\quad 一 }
\end{center}

\vspace{3pt}

\begin{enumerate}
    \item 写出命题$p\Longleftrightarrow 
 q$的真值表,其中$p,q$为任意命题. 
 \item 利用真值表,证明\textit{德摩根律}\[\left\{\begin{array}{c}
          \neg (p\lor q)\quad \Longleftrightarrow \quad \neg p\land \neg q \\
          \neg(p\land q)\quad \Longleftrightarrow \quad \neg p\lor \neg q
    \end{array}\right.\]
    \item 用逻辑符号($\forall,\,\exists$ 等)严格写出下面命题,并写出其否定形式.
    \begin{enumerate}
        \item 非空数集$X$的最小值是$m$.
        \item $f$是区间$(a,b)$上的单调增函数.
        \item $f$是区间$(a,b)$上的单调函数.
        \item $A-B:=\{a-b\mid a\in A,\,b\in B\}\neq \emptyset$
    \end{enumerate}
    \item 若$T:\mathbb{R}\to\mathbb{R}$是线性映射,证明$T$是单射当且仅当:若$T(x)=0$,则$x=0$. 
    \item 若$T:\mathbb{R}\to\mathbb{R}$是线性应映射,证明$T$是单射当且仅当$T$是满射. 
     \item 对映射$T$及其逆映射$T^{-1}$,证明有$T\circ T^{-1}=I|_{R(T)};\,\, T^{-1}\circ T=I|_{D(T)}$,其中$I$代表恒等映射,即满足$I(x)=x$的映射. 
     \item  \textit{迪利克雷(Dirichlet)函数}定义为: $D(x):=\left\{\begin{array}{c}
     1\qquad x\in \mathbb{Q}  \\
     0\qquad x\notin \mathbb{Q}
\end{array}\right.$、\quad  迪利克雷函数是否为周期函数?如果是,其最小正周期是否存在?
\item 利用欧拉公式证明三角函数的加法公式.
\item 验证$\cosh^{2}{x}-\sinh^{2}{x}=1$ 和$\cosh{(x\pm y)}=\cosh{x}\cosh{y}\pm \sinh{x}\sinh{y}$
\item 写出\textit{尖点曲线}$y^{2}=x^{3}$的一个参数方程描述.
\item 不用导数的定义,求三次曲线$y=x^{3}+2x+3$在$x=1$处的切线方程.
\item 将下列隐函数方程曲线转化为参数方程曲线,并指出参数的变化范围.
\[a)\quad 4x^{2}-4x+y^{2}+2y=0;\qquad b)\quad e^{y}+y^{3}+2x=1\]
\item 将下列曲线方程转化为极坐标方程,并指出$\theta$的变化范围.
\[a)\quad x^{2}-y^{2}=1;\qquad b)\quad (x^{2}+y^{2})^{\frac{3}{2}}=x^{2}-y^{2}\]
\item 绘制下列极坐标方程表示的曲线的图形.
\[a)\quad r=a\theta\,(a>0);\qquad b)\quad r=\tan\theta\sec\theta\]
\item 设函数$f(x)$在$\mathbb{R}$上有定义,把满足$f(x^{*})=x^{*}$ 的点$x^{*}$称为$f(x)$ 的\textit{不动点}. 证明:若$f(f(x))$有唯一不动点,则$f(x)$ 也有唯一不动点.
\end{enumerate}




















 
   
\end{document}
